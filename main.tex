% !TEX root
% !TEX program = pdflatex
% !BIB program = biber

\pdfcompresslevel 0
\pdfobjcompresslevel 0
%\special{dvipdfmx:config z 0}
\documentclass[a4paper,oneside]{amsart}
\usepackage[utf8]{inputenc}
% *%%%%%%%%%%%%%%%%%%%%%%%%%%%%%%%%%%%%%%%%%%%%%%%%
% * Fonts, symbols and notations
\let\opn\operatorname

\usepackage{mathtools,amssymb,textcomp,extarrows,bm,mleftright,stmaryrd}
\mleftright


\usepackage{mathrsfs}
\usepackage{xparse}
\ExplSyntaxOn
\NewDocumentCommand{\definealphabet}{mmmm}
 {% #1 = prefix, #2 = command, #3 = start, #4 = end
  \int_step_inline:nnn { `#3 } { `#4 }
   {
    \cs_new_protected:cpx { #1 \char_generate:nn { ##1 }{ 11 } }
     {
      \exp_not:N #2 { \char_generate:nn { ##1 } { 11 } }
     }
   }
 }
\ExplSyntaxOff
\definealphabet{bb}{\mathbb}{A}{Z}
\definealphabet{cal}{\mathcal}{A}{Z}
\definealphabet{frak}{\mathfrak}{A}{z}
\definealphabet{rm}{\mathrm}{A}{z}
\definealphabet{bf}{\mathbf}{A}{z}
\definealphabet{scr}{\mathscr}{A}{Z}
\DeclareMathOperator{\Gal}{Gal}
\DeclareMathOperator{\cycl}{cycl}
\DeclareMathOperator{\Rep}{\mathbf{Rep}}
\DeclareMathOperator{\Tr}{Tr}
\DeclareMathOperator{\Supp}{Supp}
\DeclareMathOperator{\im}{im}
\newcommand{\lto}{\longrightarrow}
\newcommand{\wtilde}[1]{\widetilde{#1}}
\let\xlto\xlongrightarrow


\usepackage{tikz-cd}
\usetikzlibrary{decorations.pathmorphing,cd}

\DeclareMathOperator{\Ker}{Ker}

%\usepackage[new]{old-arrows}

\newcommand{\Frac}{\opn{Frac}}
\newcommand{\id}{\opn{id}}

\newcommand*\dif{\mathop{}\!\mathrm{d}}
%\usepackage{standalone}
% *%%%%%%%%%%%%%%%%%%%%%%%%%%%%%%%%%%%%%%%%%%%%%%%%
% * Reference and hyperlink
\usepackage[pdfencoding=auto,psdextra]{hyperref}
\usepackage[nameinlink]{cleveref}
\Crefname{conjecture}{Conjecture}{Conjectures}
\Crefname{lemma}{Lemma}{Lemmas}
\Crefname{definition}{Definition}{Definitions}
\Crefname{remark}{Remark}{Remarks}
\Crefname{proposition}{Proposition}{Propositions}
\Crefname{corollary}{Corollary}{Corollarys}
\Crefname{equation}{}{}
\Crefname{item}{}{}
\Crefname{algorithm}{Algorithm}{Algorithms}
\Crefname{example}{Example}{Examples}
\Crefname{proof}{Proof}{Proofs}
\Crefname{condition}{Condition}{Conditions}
\Crefname{question}{Question}{Questions}
\usepackage[url=false,backend=biber,style=ext-alphabetic,hyperref=true,giveninits=true,maxbibnames=99]{biblatex}
\addbibresource{references.bib}
\usepackage{xcolor}
\makeatletter
\pdfstringdefDisableCommands{\let\HyPsd@CatcodeWarning\@gobble}
\makeatother
% *%%%%%%%%%%%%%%%%%%%%%%%%%%%%%%%%%%%%%%%%%%%%%%%%
% * Environments related
%\begingroup
\usepackage{enumitem,float,algorithm,algpseudocode,cases,adjustbox}
\usepackage[all]{xy}
\newtheorem{theorem}{Theorem}[subsection]
\newtheorem*{theorem*}{Theorem}
\newtheorem{example}[theorem]{Example}
\newtheorem{lemma}[theorem]{Lemma}
\newtheorem{remark}[theorem]{Remark}
\newtheorem{proposition}[theorem]{Proposition}
\newtheorem*{proposition*}{Proposition}
\newtheorem{definition}[theorem]{Definition}
\newtheorem{conjecture}[theorem]{Conjecture}
\newtheorem{corollary}[theorem]{Corollary}
\newtheorem{question}[theorem]{Question}
\renewcommand{\theequation}{\alph{equation}}
\numberwithin{equation}{section}
\numberwithin{figure}{section}
\newlist{propenum}{enumerate}{1}
\setlist[propenum]{label=(\arabic*), ref=\theproposition~(\arabic*)}
\crefalias{propenumi}{proposition}
\newlist{lemenum}{enumerate}{1}
\setlist[lemenum]{label=(\arabic*), ref=\thelemma~(\arabic*)}
\crefalias{lemenumi}{lemma}
\newlist{thmenum}{enumerate}{1}
\setlist[thmenum]{label=(\arabic*), ref=\thetheorem~(\arabic*)}
\crefalias{thmenumi}{theorem}
\Crefformat{enumi}{#2\textup{(#1)}#3}
\usepackage{comment}
%\usepackage{float,algorithm,algpseudocode}
\algnewcommand\algorithmicinput{\textbf{INPUT:}}
\algnewcommand\INPUT{\item[\algorithmicinput]}
\algnewcommand\algorithmicoutput{\textbf{OUTPUT:}}
\algnewcommand\OUTPUT{\item[\algorithmicoutput]}
%\endgroup
% *%%%%%%%%%%%%%%%%%%%%%%%%%%%%%%%%%%%%%%%%%%%%%%%%
% * Typesetting
\makeatletter
\def\@tocline#1#2#3#4#5#6#7{\relax
  \ifnum #1>\c@tocdepth % then omit
  \else
    \par \addpenalty\@secpenalty\addvspace{#2}%
    \begingroup \hyphenpenalty\@M
    \@ifempty{#4}{%
      \@tempdima\csname r@tocindent\number#1\endcsname\relax
    }{%
      \@tempdima#4\relax
    }%
    \parindent\z@ \leftskip#3\relax \advance\leftskip\@tempdima\relax
    \rightskip\@pnumwidth plus4em \parfillskip-\@pnumwidth
    #5\leavevmode\hskip-\@tempdima
      \ifcase #1
       \or\or \hskip 1em \or \hskip 2em \else \hskip 3em \fi%
      #6\nobreak\relax
    \hfill\hbox to\@pnumwidth{\@tocpagenum{#7}}\par% <---- \dotfill -> \hfill
    \nobreak
    \endgroup
  \fi}
\makeatother
\usepackage{microtype}
\usepackage{geometry}
\allowdisplaybreaks
% *%%%%%%%%%%%%%%%%%%%%%%%%%%%%%%%%%%%%%%%%%%%%%%%%
% * Temporary
\usepackage{CJKutf8}
\newcommand{\Chinese}[1]{\begin{CJK*}{UTF8}{gbsn}#1\end{CJK*}}
%\usepackage[scheme=plain]{ctex}
%\usepackage[notcite]{showkeys}

\usepackage{todonotes}
%\usepackage[tikz]{bclogo}
% *%%%%%%%%%%%%%%%%%%%%%%%%%%%%%%%%%%%%%%%%%%%%%%%%
%%TO BE REMOVED FROM THE FINAL
\usepackage[color,notcite, notref]{showkeys} % print labels in the pdf
\definecolor{labelkey}{rgb}{1,0,0}
%\lineskip .2 cm %% for readability
%\parskip 0.2cm
\newcommand{\shanwen}[1]{\textcolor{magenta}{[Shanwen: #1]}}
\newcommand{\yijun}[1]{\textcolor{blue}{[Yijun: #1]}}
\makeatletter
  \SK@def\Cref#1{\SK@\SK@@ref{#1}\SK@Cref{#1}}%
\makeatother

\title{Iwasawa cohomology of $p$-adic Galois representations and explicit reciprocity law}


\usepackage{orcidlink}
\author{Yijun Yuan\textsuperscript{\orcidlink{0000-0001-6571-6980}} 2022311357}
\address{Yau Mathematical Sciences Center, Tsinghua University, Beijing 100084, China}
\email{941201yuan@gmail.com}
\begin{document}
\maketitle
Throughout this note, $p\geq 3$ is a prime number. Let $K$ be a $p$-adic field. Let $(\zeta_{p^n})_{k\geq 1}$ be a system of norm compatible system of $p^n$-th root of unity in $\overline{\bbQ}_p$ and let $K_n^{\opn{cycl}}=K(\zeta_{p^n})$, $K^{\opn{cycl}}=\bigcup_n K_n$ be the cyclotomic extension over $K$.

For any field $F$, denote by $\scrG_F=\opn{Gal}(\overline{F}/F)$ the absolute Galois group of $F$. In particular, denote by $H_K=\scrG_{K^{\opn{cycl}}}$ and $\Gamma_K=\scrG_K/H_K=\opn{Gal}(K^{\opn{cycl}}/K)$.
\section{Theory of $(\varphi,\Gamma)$-modules}
The theory of \'etale $(\varphi,\Gamma)$-modules, which was introduced by Fontaine in the early 1990s, is used to study the $p$-adic Galois representations. These are modules over certain period rings, equipped with the semi-linear action of a Frobenius map and $\Gamma_K$ which commute with each other.

\subsection{The field of norms and its lift to characteristic $0$}
The construction of $(\varphi,\Gamma)$-modules depends deeply on the so-called field of norms (corps des norms), which was introduced by Fontaine and Wintenberger in the late 1970s (cf. \cite{fontaine_extensions_1979,fontaine_corps_1979,wintenberger_corps_1983}). Given a so-called strictly arithmetically profinite (sAPF) extension (cf. \cite[D\'efinitions 1.2.1]{wintenberger_corps_1983}) of $p$-adic field $K_\infty/K$, the field of norms associated with this extension, which is a field of characteristic $p$, has the same absolute Galois group as $K_\infty$. This theory also motivates Scholze's theory of perfectoid spaces (cf. \cite[Theorem 1.1]{scholze_perfectoid_2012}).

In this notes, we do not need the full power of sAPF extensions (which we will not recall the precise definition), so we are going to specialize Fontaine-Wintenberger's results to the cyclotomic extension $K^{\opn{cycl}}/K$ while keeping the fact in mind that it is sAPF\footnote{This is based on a result of Sen (cf. \cite{sen_ramification_1972}): if a totally ramified $p$-adic normal extension $K_\infty/K$ has a $p$-adic Lie group as its Galois group, then it is sAPF.}.
\begin{definition}[{\cite[Section 2.1.1]{wintenberger_corps_1983}}]
	Let $$X_K(K^{\cycl})\coloneqq\varprojlim_n \left(K_n^{\opn{cycl}}\right)^\times\cup\{0\},$$
	where the transition map $\calN_n:\left(K_{n+1}^{\opn{cycl}}\right)^\times\lto \left(K_n^{\opn{cycl}}\right)^\times$ is the norm map. We call it the field of norms of the extension $K^{\opn{cycl}}/K$.
\end{definition}
\begin{theorem}[Fontaine-Wintenberger]\leavevmode
	\begin{enumerate}
		\item (cf. \cite[Th\'eor\`eme 2.1.3]{wintenberger_corps_1983}) By defining [] and [], $X_K(K^{\opn{cycl}})$ is a complete discrete valued field with valuation given by
		      $$v((x_n)_{n\geq 1})=[v_p(K^\times):\bbZ]\cdot v_p(x_0).$$
		\item (cf. \cite[Corollaire 3.2.3]{wintenberger_corps_1983}) The absolute Galois group of $X_K(K^{\opn{cycl}})$ is canonically isomorphic to $H_K$.
		\item (cf. \cite[Proposition 4.2.1]{wintenberger_corps_1983}) The field $X_K(K^{\opn{cycl}})$ embeds continuously into the perfectoid field $\widetilde{\bfE}=\bbC_p^\flat\coloneqq \varprojlim_{x\mapsto x^p}\bbC_p$ by the formula [].
		\item The field $X_K(K^{\opn{cycl}})$ is naturally equipped with the action of Frobenius map $x\mapsto x^p$ and the Galois group $\Gamma_K$.
	\end{enumerate}
\end{theorem}
\begin{definition}
	Let $\bfE_K$ be the image of $X_K(K^{\opn{cycl}})$ in $\widetilde{\bfE}$. Set $\bfE=\bigcup_{[K:\bbQ_p]<\infty}\bfE_K$ and $\widetilde{\bfE}_K=\widetilde{\bfE}^{H_K}$.
\end{definition}
\begin{remark}
	With easy calculation, one can see that the element $\varepsilon-1$, where $\varepsilon=(\zeta_{p^n})_{n\geq 1}\in \varprojlim_{x\mapsto x^p}\calO_{\bbC_p}$, is a uniformizer of $\bfE_{\bbQ_p}$.
\end{remark}
Then one has
\begin{proposition}\leavevmode
	\begin{enumerate}
		\item $\bfE$ is the separable closure of $\bfE_{\bbQ_p}$ and is stable under the action of $\scrG_{\bbQ_p}$.
		\item (cf. \cite[Corollaire 4.3.4]{wintenberger_corps_1983}) $\widetilde{\bfE}$ (resp. $\widetilde{\bfE}_K$) is the completed perfect closure of $\bfE$ (resp. $\bfE_K$).
	\end{enumerate}
\end{proposition}

Now we can lift these period rings to characteristic $0$:
\begin{definition}
	Let $\widetilde{\bfA}=W(\widetilde{\bfE})$ be the ring of Witt vectors over $\widetilde{\bfE}$. Let $\bfA_K=\opn{Cohen}(\bfE_K)$ (resp. $\bfA=\opn{Cohen}(\bfE)$) be the Cohen ring\footnote{We refer to \cite[Chapitre IX]{bourbakiAlgebreCommutative2006} for the detailed construction of the Wiit vectors and the Cohen ring.} over $\bfE_K$ (resp. $\bfE$). Denote by $\widetilde{\bfB}$ (resp. $\bfB_K$, $\bfB$) the fraction field of $\widetilde{\bfA}$ (resp. $\bfA_K$, $\bfA$).
\end{definition}

We have the following properties of these period rings of characteristic $0$:
\begin{proposition}\leavevmode
	\begin{enumerate}
		\item The field $\bfB_K$ is a discrete-valued field with valuation ring $\bfA_K$ and residue field $\bfE_K$.
		\item The field $\bfB$ is the completed maximal unramified extension of $\bfB_{\bbQ_p}$, with ring of integer $\bfA$ and residue field $\bfE$.
		\item One has
		      $$\bfA_{\bbQ_p}=\widehat{\bbZ_p\llbracket\pi\rrbracket[\frac{1}{\pi}]},$$
		      where $\pi\coloneqq [\varepsilon]-1$ and the hat is taking the $p$-adic completion.
		\item The action of $\scrG_{\bbQ_p}$ on $\bfA_{\bbQ_p}$ is given by
		      $$g\cdot\pi=(1+\pi)^{\chi(g)}-1,$$ where $\chi:\scrG_{\bbQ_p}\lto \bbZ_p^\times$ is the cyclotomic character. The action of the Frobenius map $\varphi$ on $\bfA_{\bbQ_p}$ is given by
		      $$\varphi(\pi)=(1+\pi)^p-1.$$
		\item One has $\bfA^{H_K}=\bfA_K$, $\bfB^{H_K}=\bfB_K$.
		\item The field $\widetilde{\bfB}$ (resp. $\widetilde{\bfB}_K$) is the completed $\varphi$-perfection of $\bfB$ (resp. $\bfB_K$).
	\end{enumerate}
\end{proposition}
\subsection{\'Etale $(\varphi,\Gamma)$-modules and Fontaine's equivalence of categories}
\begin{definition}
	\begin{enumerate}
		\item A $\varphi$-module over $\bfA_K$ (resp. $\bfB_K$) is a finitely generated $\bfA_K$-module (resp. $\bfB_K$-vector space) equipped with a semi-linear continuous action of the Frobenius map $\varphi$.
		\item A $\varphi$-module $D$ over $\bfA_K$ is called \'etale if $\varphi(D)$ generates $D$ as an $\bfA_K$-module.
		\item A $\varphi$-module $D$ over $\bfB_K$ is called \'etale if it has an $\bfA_K$-lattice which is \'etale.
	\end{enumerate}
\end{definition}
\begin{theorem}
	The functor $\calD$:
	$$\{\text{finite dimensional }\bbZ_p\text{-representations of }\scrG_K\}\xlto{\calD} \{\text{\'etale } \varphi\text{-modules over }\bfA_K\},$$
	$$V\longmapsto (\bfA\otimes_{\bbZ_p} V)^{H_K};$$
	is an equivalence of categories, with quasi-inverse $\calV$ given by
	$$\{\text{\'etale } \varphi\text{-modules over }\bfA_K\}\xlto{\calV} \{\text{finite dimensional }\bbZ_p\text{-representations of }\scrG_K\},$$
	$$D\longmapsto (\bfA\otimes_{\bfA_K} D)^{\varphi=1}.$$
\end{theorem}
\begin{remark}
	By replacing $\bfA$ (resp. $\bfA_K$) by $\bfB$ (resp. $\bfB_K$), one can also define the equivalence of categories between finite-dimensional $\bbQ_p$-representations of $\scrG_K$ and \'etale $\varphi$-modules over $\bfB_K$.
\end{remark}
\section{Galois cohomology and Iwasawa cohomology}
\section{Explicit reciprocity law}

\printbibliography
\end{document}