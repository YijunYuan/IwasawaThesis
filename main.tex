% !TEX root
% !TEX program = pdflatex
% !BIB program = biber

\pdfcompresslevel 0
\pdfobjcompresslevel 0
%\special{dvipdfmx:config z 0}
\documentclass[a4paper,oneside]{amsart}
\usepackage[utf8]{inputenc}
% *%%%%%%%%%%%%%%%%%%%%%%%%%%%%%%%%%%%%%%%%%%%%%%%%
% * Fonts, symbols and notations
\let\opn\operatorname

\usepackage{mathtools,amssymb,textcomp,extarrows,bm,mleftright,graphicx,stmaryrd,scalerel}
\mleftright

\makeatletter
\AtBeginDocument{%
 \let\glb@currsize\relax
}
\makeatother

\usepackage{iftex}
\ifpdftex
\usepackage[T1]{fontenc}
\pdfmapline{+tamafont < tamafont.ttf <T1-WGL4.enc}
\DeclareMathAlphabet{\mathtama}{T1}{tamafont}{m}{n}
\newcommand{\tama}{\mathtama{0}}
\fi
\ifxetex
\usepackage{fontspec}
\newfontfamily{\tamafont}{tamafont}[
  NFSSFamily=tamafont,
  Path=./,
  Extension=.ttf
]
\DeclareSymbolFont{tmf}{TU}{tamafont}{m}{n}
\DeclareMathSymbol{\tama}{0}{tmf}{"30}
\fi
\usepackage{mathrsfs}
\usepackage{xparse}
\ExplSyntaxOn
\NewDocumentCommand{\definealphabet}{mmmm}
 {% #1 = prefix, #2 = command, #3 = start, #4 = end
  \int_step_inline:nnn { `#3 } { `#4 }
   {
    \cs_new_protected:cpx { #1 \char_generate:nn { ##1 }{ 11 } }
     {
      \exp_not:N #2 { \char_generate:nn { ##1 } { 11 } }
     }
   }
 }
 \makeatletter\providecommand*\xLongrightleftharpoons[2][]{\mathrel{%
  \raise.22ex\hbox{%
    $\ext@arrow 9999\MT_rightharpoonup_fill:{\phantom{\Longleftrightarrow}}{#2}$}%
  \setbox0=\hbox{%
    $\ext@arrow 9999\MT_leftharpoondown_fill:{#1}{\phantom{\Longleftrightarrow}}$}%
  \kern-\wd0 \lower.22ex\box0}}
\ExplSyntaxOff
\definealphabet{bb}{\mathbb}{A}{Z}
\definealphabet{cal}{\mathcal}{A}{Z}
\definealphabet{frak}{\mathfrak}{A}{z}
\definealphabet{rm}{\mathrm}{A}{z}
\definealphabet{bf}{\mathbf}{A}{Z}
\definealphabet{scr}{\mathscr}{A}{Z}
\DeclareMathOperator{\Gal}{Gal}
\DeclareMathOperator{\proj}{proj}
\DeclareMathOperator{\cycl}{cycl}
\DeclareMathOperator{\Kum}{Kum}
\DeclareMathOperator{\FT}{FT}
\DeclareMathOperator{\sep}{sep}
\DeclareMathOperator{\Rep}{\mathbf{Rep}}
\DeclareMathOperator{\Tr}{Tr}
\DeclareMathOperator{\Supp}{Supp}
\DeclareMathOperator{\im}{im}
\newcommand{\lto}{\longrightarrow}
\newcommand{\wtilde}[1]{\widetilde{#1}}
\let\xlto\xlongrightarrow


\usepackage{environ,ifdraft}
\ifdraft{
    \NewEnviron{tikzcd}{\text{\includegraphics[draft,width=3cm]{example-image}}}
}{
    \usepackage{tikz}
\usetikzlibrary{decorations.pathmorphing,cd}
}


\usepackage{dsfont}
\newcommand{\bbone}{\mathds{1}}

\DeclareMathOperator{\Ker}{Ker}

\usepackage[new]{old-arrows}

\newcommand{\Frac}{\opn{Frac}}
\newcommand{\id}{\opn{id}}
%\usepackage{standalone}
% *%%%%%%%%%%%%%%%%%%%%%%%%%%%%%%%%%%%%%%%%%%%%%%%%
% * Reference and hyperlink
\usepackage[pdfencoding=auto,psdextra]{hyperref}
\usepackage[nameinlink]{cleveref}
\Crefname{conjecture}{Conjecture}{Conjectures}
\Crefname{lemma}{Lemma}{Lemmas}
\Crefname{definition}{Definition}{Definitions}
\Crefname{remark}{Remark}{Remarks}
\Crefname{proposition}{Proposition}{Propositions}
\Crefname{corollary}{Corollary}{Corollarys}
\Crefname{equation}{}{}
\Crefname{item}{}{}
\Crefname{algorithm}{Algorithm}{Algorithms}
\Crefname{example}{Example}{Examples}
\Crefname{proof}{Proof}{Proofs}
\Crefname{condition}{Condition}{Conditions}
\Crefname{question}{Question}{Questions}
\usepackage[url=false,backend=biber,style=ext-alphabetic,hyperref=true,giveninits=true,maxbibnames=99]{biblatex}
\addbibresource{references.bib}
\usepackage{xcolor}
\makeatletter
\pdfstringdefDisableCommands{\let\HyPsd@CatcodeWarning\@gobble}
\makeatother
% *%%%%%%%%%%%%%%%%%%%%%%%%%%%%%%%%%%%%%%%%%%%%%%%%
% * Environments related
%\begingroup
\usepackage{enumitem,float,algorithm,algpseudocode,cases,adjustbox}
\usepackage[all]{xy}
\newtheorem{theorem}{Theorem}[subsection]
\newtheorem*{theorem*}{Theorem}
\newtheorem{example}[theorem]{Example}
\newtheorem{lemma}[theorem]{Lemma}
\newtheorem{remark}[theorem]{Remark}
\newtheorem{proposition}[theorem]{Proposition}
\newtheorem*{proposition*}{Proposition}
\newtheorem{definition}[theorem]{Definition}
\newtheorem{conjecture}[theorem]{Conjecture}
\newtheorem{corollary}[theorem]{Corollary}
\newtheorem{question}[theorem]{Question}
\renewcommand{\theequation}{\alph{equation}}
\numberwithin{equation}{section}
\numberwithin{figure}{section}
\newlist{propenum}{enumerate}{1}
\setlist[propenum]{label=(\arabic*), ref=\theproposition~(\arabic*)}
\crefalias{propenumi}{proposition}
\newlist{lemenum}{enumerate}{1}
\setlist[lemenum]{label=(\arabic*), ref=\thelemma~(\arabic*)}
\crefalias{lemenumi}{lemma}
\newlist{thmenum}{enumerate}{1}
\setlist[thmenum]{label=(\arabic*), ref=\thetheorem~(\arabic*)}
\crefalias{thmenumi}{theorem}
\Crefformat{enumi}{#2\textup{(#1)}#3}
\usepackage{comment}
%\usepackage{float,algorithm,algpseudocode}
\algnewcommand\algorithmicinput{\textbf{INPUT:}}
\algnewcommand\INPUT{\item[\algorithmicinput]}
\algnewcommand\algorithmicoutput{\textbf{OUTPUT:}}
\algnewcommand\OUTPUT{\item[\algorithmicoutput]}
%\endgroup
% *%%%%%%%%%%%%%%%%%%%%%%%%%%%%%%%%%%%%%%%%%%%%%%%%
% * Typesetting
\makeatletter
\def\@tocline#1#2#3#4#5#6#7{\relax
  \ifnum #1>\c@tocdepth % then omit
  \else
    \par \addpenalty\@secpenalty\addvspace{#2}%
    \begingroup \hyphenpenalty\@M
    \@ifempty{#4}{%
      \@tempdima\csname r@tocindent\number#1\endcsname\relax
    }{%
      \@tempdima#4\relax
    }%
    \parindent\z@ \leftskip#3\relax \advance\leftskip\@tempdima\relax
    \rightskip\@pnumwidth plus4em \parfillskip-\@pnumwidth
    #5\leavevmode\hskip-\@tempdima
      \ifcase #1
       \or\or \hskip 1em \or \hskip 2em \else \hskip 3em \fi%
      #6\nobreak\relax
    \hfill\hbox to\@pnumwidth{\@tocpagenum{#7}}\par% <---- \dotfill -> \hfill
    \nobreak
    \endgroup
  \fi}
\makeatother
\usepackage{microtype}
\usepackage{geometry}
\allowdisplaybreaks
% *%%%%%%%%%%%%%%%%%%%%%%%%%%%%%%%%%%%%%%%%%%%%%%%%
% * Temporary
\usepackage{CJKutf8}
\newcommand{\Chinese}[1]{\begin{CJK*}{UTF8}{gbsn}#1\end{CJK*}}
%\usepackage[scheme=plain]{ctex}
%\usepackage[notcite]{showkeys}

\usepackage{todonotes}
%\usepackage[tikz]{bclogo}
% *%%%%%%%%%%%%%%%%%%%%%%%%%%%%%%%%%%%%%%%%%%%%%%%%
%%TO BE REMOVED FROM THE FINAL
\usepackage[color,notcite, notref]{showkeys} % print labels in the pdf
\definecolor{labelkey}{rgb}{1,0,0}
%\lineskip .2 cm %% for readability
%\parskip 0.2cm
\newcommand{\shanwen}[1]{\textcolor{magenta}{[Shanwen: #1]}}
\newcommand{\yijun}[1]{\textcolor{blue}{[Yijun: #1]}}
\makeatletter
  \SK@def\Cref#1{\SK@\SK@@ref{#1}\SK@Cref{#1}}%
\makeatother

\title{Iwasawa cohomology of $p$-adic Galois representations and explicit reciprocity law}


\usepackage{orcidlink}
\author{Yijun Yuan\orcidlink{0000-0001-6571-6980}}
\address{Yau Mathematical Sciences Center, Tsinghua University, Beijing 100084, China}
\email{941201yuan@gmail.com}
\begin{document}
\maketitle
Throughout this notes, $p\geq 3$ is a prime number. Let $K$ be a $p$-adic field. Let $(\zeta_{p^n})_{k\geq 1}$ be a system of norm compatible system of $p^n$-th root of unity in $\overline{\bbQ}_p$ and let $K_n^{\opn{cycl}}=K(\zeta_{p^n})$, $K^{\opn{cycl}}=\bigcup_n K_n$ be the cyclotomic extension over $K$.

For any field $F$, denote by $\scrG_F=\opn{Gal}(\overline{F}/F)$ the absolute Galois group of $F$. In particular, denote by $H_K=\scrG_{K^{\opn{cycl}}}$ and $\Gamma_K=\scrG_K/H_K=\opn{Gal}(K^{\opn{cycl}}/K)$.
\section{Theory of $(\varphi,\Gamma)$-modules}
The theory of \'etale $(\varphi,\Gamma)$-modules, which was introduced by Fontaine in the early 1990's, is used to study the $p$-adic Galois representations. These are modules over certain period rings, equipped with the semi-linear action of a Frobenius map and $\Gamma_K$ which commutes with each other.

\subsection{The field of norms and its lift to characteristic $0$}
The construction of $(\varphi,\Gamma)$-modules depends deeply on the so called field of norms (les corps des norms), which was introduced by Fontaine and Wintenberger in 1980's. It is a wide family of fields of formal Laurent series in characteristic $p$, whose absolute Galois groups are canonically isomorphic to those of the so called strictly arithmetically profinite (sAPF) extensions. This theory also motivates Scholze's thoery of perfectoid spaces.

In this notes, we do not need the full power of sAPF extensions, so we are going to specialize Fontaine-Wintenberger's results to the cyclotomic extension $K^{\opn{cycl}}/K$ with keeping the fact in mind that it is sAPF\footnote{This is based on a result of Sen: if a totally ramified $p$-adic normal extension $K_\infty/K$ has a $p$-adic Lie group as the Galois group, then it is sAPF.}.
\begin{definition}
    Let $$X_K(K^{\cycl})\coloneqq\varprojlim_n \left(K_n^{\opn{cycl}}\right)^\times\cup\{0\},$$
    where the transition map $\calN_n:\left(K_{n+1}^{\opn{cycl}}\right)^\times\lto \left(K_n^{\opn{cycl}}\right)^\times$ is the norm map. We call it the field of norms of the extension $K^{\opn{cycl}}/K$.
\end{definition}
\begin{theorem}[{\cite{wintenberger_corps_1983}}]\leavevmode
\begin{enumerate}
    \item By defining [] and [], $X_K(K^{\opn{cycl}})$ is a complete discrete valued field with valuation given by
    $$v((x_n)_{n\geq 1})=[v_p(K^\times):\bbZ]\cdot v_p(x_0).$$
    \item The absolute Galois group of $X_K(K^{\opn{cycl}})$ is canonically isomorphic to $H_K$.
    \item The field $X_K(K^{\opn{cycl}})$ embeds continuously into the perfectoid field $\widetilde{\bfE}=\bbC_p^\flat\coloneqq \varprojlim_{x\mapsto x^p}\bbC_p$ by the formula [].
    \item The field $X_K(K^{\opn{cycl}})$ is naturally equipped with the action of Frobenius map $x\mapsto x^p$ and the Galois group $\Gamma_K$.
\end{enumerate}
\end{theorem}
\begin{definition}
    Let $\bfE_K$ be the image of $X_K(K^{\opn{cycl}})$ in $\widetilde{\bfE}$. Set $\bfE=\bigcup_{[K:\bbQ_p]<\infty}\bfE_K$ and $\widetilde{\bfE}_K=\widetilde{\bfE}^{H_K}$.
\end{definition}
Then one has
\begin{theorem}\leavevmode
    \begin{enumerate}
        \item $\bfE$ is the separable closure of $\bfE_{\bbQ_p}$ and is stable under the action of $\scrG_{\bbQ_p}$.
        \item $\widetilde{\bfE}$ (resp. $\widetilde{\bfE}_K$) is the completed perfect closure of $\bfE$ (resp. $\bfE_K$).
    \end{enumerate}
\end{theorem}

Now we can lift these period rings to characteristic $0$:
\begin{definition}
    Let $\widetilde{\bfA}=W(\widetilde{\bfE})$ be the ring of Witt vectors over $\widetilde{\bfE}$. Let $\bfA_K=\opn{Cohen}(\bfE_K)$ (resp. $\bfA=\opn{Cohen}(\bfE)$) be the Cohen ring over $\bfE_K$ (resp. $\bfE$). Denote by $\widetilde{\bfB}$ (resp. $\bfB_K$, $\bfB$) the fraction field of $\widetilde{\bfA}$ (resp. $\bfA_K$, $\bfA$).
\end{definition}

By the functoriality of Witt vectors and Cohen rings, we have the following properties:
\begin{lemma}\leavevmode
    \begin{enumerate}
        \item 
    \end{enumerate}
\end{lemma}
\section{Galois cohomology and Iwasawa cohomology}
\section{Explicit reciprocity law}
\end{document}