% !TEX root
% !TEX program = xelatex
% !BIB program = biber

%\special{dvipdfmx:config z 0}
\documentclass[a4paper,oneside]{amsart}
\usepackage[scheme=plain]{ctex}
% *%%%%%%%%%%%%%%%%%%%%%%%%%%%%%%%%%%%%%%%%%%%%%%%%
% * Fonts, symbols and notations
\let\opn\operatorname

\usepackage{mathtools,amssymb,textcomp,extarrows,bm,mleftright,stmaryrd}
\mleftright


\usepackage{mathrsfs}
\usepackage{xparse}
\ExplSyntaxOn
\NewDocumentCommand{\definealphabet}{mmmm}
 {% #1 = prefix, #2 = command, #3 = start, #4 = end
  \int_step_inline:nnn { `#3 } { `#4 }
   {
    \cs_new_protected:cpx { #1 \char_generate:nn { ##1 }{ 11 } }
     {
      \exp_not:N #2 { \char_generate:nn { ##1 } { 11 } }
     }
   }
 }
\ExplSyntaxOff
\definealphabet{bb}{\mathbb}{A}{Z}
\definealphabet{cal}{\mathcal}{A}{Z}
\definealphabet{frak}{\mathfrak}{A}{z}
\definealphabet{rm}{\mathrm}{A}{z}
\definealphabet{bf}{\mathbf}{A}{z}
\definealphabet{scr}{\mathscr}{A}{Z}
\DeclareMathOperator{\Gal}{Gal}
\DeclareMathOperator{\cycl}{cycl}
\DeclareMathOperator{\Rep}{\mathbf{Rep}}
\DeclareMathOperator{\Tr}{Tr}
\DeclareMathOperator{\Supp}{Supp}
\DeclareMathOperator{\im}{im}
\newcommand{\lto}{\longrightarrow}
\newcommand{\wtilde}[1]{\widetilde{#1}}
\let\xlto\xlongrightarrow


\usepackage{tikz-cd}
\usetikzlibrary{decorations.pathmorphing,cd}

\DeclareMathOperator{\Ker}{Ker}

%\usepackage[new]{old-arrows}

\newcommand{\Frac}{\opn{Frac}}
\newcommand{\id}{\opn{id}}

\newcommand*\dif{\mathop{}\!\mathrm{d}}
%\usepackage{standalone}
% *%%%%%%%%%%%%%%%%%%%%%%%%%%%%%%%%%%%%%%%%%%%%%%%%
% * Reference and hyperlink
\usepackage[pdfencoding=auto,psdextra]{hyperref}
\usepackage[nameinlink]{cleveref}
\Crefname{conjecture}{Conjecture}{Conjectures}
\Crefname{lemma}{Lemma}{Lemmas}
\Crefname{definition}{Definition}{Definitions}
\Crefname{remark}{Remark}{Remarks}
\Crefname{proposition}{Proposition}{Propositions}
\Crefname{corollary}{Corollary}{Corollarys}
\Crefname{equation}{}{}
\Crefname{item}{}{}
\Crefname{algorithm}{Algorithm}{Algorithms}
\Crefname{example}{Example}{Examples}
\Crefname{proof}{Proof}{Proofs}
\Crefname{condition}{Condition}{Conditions}
\Crefname{question}{Question}{Questions}
\usepackage[url=false,backend=biber,style=ext-alphabetic,hyperref=true,giveninits=true,maxbibnames=99]{biblatex}
\addbibresource{references.bib}
\usepackage{xcolor}
\makeatletter
\pdfstringdefDisableCommands{\let\HyPsd@CatcodeWarning\@gobble}
\makeatother
% *%%%%%%%%%%%%%%%%%%%%%%%%%%%%%%%%%%%%%%%%%%%%%%%%
% * Environments related
%\begingroup
\usepackage{enumitem,float,algorithm,algpseudocode,cases,adjustbox}
\usepackage[all]{xy}
\newtheorem{theorem}{Theorem}[section]
\newtheorem*{theorem*}{Theorem}
\newtheorem{example}[theorem]{Example}
\newtheorem{lemma}[theorem]{Lemma}
\newtheorem{remark}[theorem]{Remark}
\newtheorem{proposition}[theorem]{Proposition}
\newtheorem*{proposition*}{Proposition}
\newtheorem{definition}[theorem]{Definition}
\newtheorem{conjecture}[theorem]{Conjecture}
\newtheorem{corollary}[theorem]{Corollary}
\newtheorem{question}[theorem]{Question}
\renewcommand{\theequation}{\alph{equation}}
\numberwithin{equation}{section}
\numberwithin{figure}{section}
\newlist{propenum}{enumerate}{1}
\setlist[propenum]{label=(\arabic*), ref=\theproposition~(\arabic*)}
\crefalias{propenumi}{proposition}
\newlist{lemenum}{enumerate}{1}
\setlist[lemenum]{label=(\arabic*), ref=\thelemma~(\arabic*)}
\crefalias{lemenumi}{lemma}
\newlist{thmenum}{enumerate}{1}
\setlist[thmenum]{label=(\arabic*), ref=\thetheorem~(\arabic*)}
\crefalias{thmenumi}{theorem}
\Crefformat{enumi}{#2\textup{(#1)}#3}
\usepackage{comment}
%\usepackage{float,algorithm,algpseudocode}
\algnewcommand\algorithmicinput{\textbf{INPUT:}}
\algnewcommand\INPUT{\item[\algorithmicinput]}
\algnewcommand\algorithmicoutput{\textbf{OUTPUT:}}
\algnewcommand\OUTPUT{\item[\algorithmicoutput]}
%\endgroup
% *%%%%%%%%%%%%%%%%%%%%%%%%%%%%%%%%%%%%%%%%%%%%%%%%
% * Typesetting
\makeatletter
\def\@tocline#1#2#3#4#5#6#7{\relax
  \ifnum #1>\c@tocdepth % then omit
  \else
    \par \addpenalty\@secpenalty\addvspace{#2}%
    \begingroup \hyphenpenalty\@M
    \@ifempty{#4}{%
      \@tempdima\csname r@tocindent\number#1\endcsname\relax
    }{%
      \@tempdima#4\relax
    }%
    \parindent\z@ \leftskip#3\relax \advance\leftskip\@tempdima\relax
    \rightskip\@pnumwidth plus4em \parfillskip-\@pnumwidth
    #5\leavevmode\hskip-\@tempdima
      \ifcase #1
       \or\or \hskip 1em \or \hskip 2em \else \hskip 3em \fi%
      #6\nobreak\relax
    \hfill\hbox to\@pnumwidth{\@tocpagenum{#7}}\par% <---- \dotfill -> \hfill
    \nobreak
    \endgroup
  \fi}
\makeatother
\usepackage{microtype}
\usepackage{geometry}
\allowdisplaybreaks
% *%%%%%%%%%%%%%%%%%%%%%%%%%%%%%%%%%%%%%%%%%%%%%%%%
% * Temporary
%\usepackage[scheme=plain]{ctex}
%\usepackage[notcite]{showkeys}

\usepackage{todonotes}
%\usepackage[tikz]{bclogo}
% *%%%%%%%%%%%%%%%%%%%%%%%%%%%%%%%%%%%%%%%%%%%%%%%%
%%TO BE REMOVED FROM THE FINAL
\usepackage[color,notcite, notref]{showkeys} % print labels in the pdf
\definecolor{labelkey}{rgb}{1,0,0}
%\lineskip .2 cm %% for readability
%\parskip 0.2cm
\newcommand{\shanwen}[1]{\textcolor{magenta}{[Shanwen: #1]}}
\newcommand{\yijun}[1]{\textcolor{blue}{[Yijun: #1]}}
\makeatletter
  \SK@def\Cref#1{\SK@\SK@@ref{#1}\SK@Cref{#1}}%
\makeatother

\title{Iwasawa cohomology of $p$-adic Galois representations and explicit reciprocity law}


\usepackage{orcidlink}
\author{袁轶君\textsuperscript{\orcidlink{0000-0001-6571-6980}} 2022311357}
\address{Yau Mathematical Sciences Center, Tsinghua University, Beijing 100084, China}
\email{941201yuan@gmail.com}
\begin{document}
\maketitle
\tableofcontents
Throughout this note, $p\geq 3$ is a prime number. Let $K$ be a $p$-adic field. Let $(\zeta_{p^n})_{k\geq 1}$ be a system of norm compatible system of $p^n$-th root of unity in $\overline{\bbQ}_p$ and let $K_n^{\opn{cycl}}=K(\zeta_{p^n})$, $K^{\opn{cycl}}=\bigcup_n K_n^{\opn{cycl}}$ be the cyclotomic extension over $K$.

For any field $F$, denote by $\scrG_F=\opn{Gal}(\overline{F}/F)$ the absolute Galois group of $F$. In particular, denote by $H_K=\scrG_{K^{\opn{cycl}}}$ and $\Gamma_K=\scrG_K/H_K=\opn{Gal}(K^{\opn{cycl}}/K)$.
\section{Theory of $(\varphi,\Gamma)$-modules}
The theory of \'etale $(\varphi,\Gamma)$-modules, which was introduced by Fontaine in the early 1990s, is used to study the $p$-adic Galois representations. These are modules over certain period rings, equipped with the semi-linear action of a Frobenius map and $\Gamma_K$ which commute with each other.

\subsection{The field of norms and its lift to characteristic $0$}
The construction of $(\varphi,\Gamma)$-modules depends deeply on the so-called field of norms (corps des norms), which was introduced by Fontaine and Wintenberger in the late 1970s (cf. \cite{fontaine_extensions_1979,fontaine_corps_1979,wintenberger_corps_1983}). Given a so-called strictly arithmetically profinite (sAPF) extension (cf. \cite[D\'efinitions 1.2.1]{wintenberger_corps_1983}) of $p$-adic field $K_\infty/K$, the field of norms associated with this extension, which is a field of characteristic $p$, has the same absolute Galois group as $K_\infty$. This theory also motivates Scholze's theory of perfectoid spaces (cf. \cite[Theorem 1.1]{scholze_perfectoid_2012}).

In this notes, we do not need the full power of sAPF extensions (which we will not recall the precise definition), so we are going to specialize Fontaine-Wintenberger's results to the cyclotomic extension $K^{\opn{cycl}}/K$ while keeping the fact in mind that it is sAPF\footnote{This is based on a result of Sen (cf. \cite{sen_ramification_1972}): if a totally ramified $p$-adic normal extension $K_\infty/K$ has a $p$-adic Lie group as its Galois group, then it is sAPF.}.
\begin{definition}[{\cite[Section 2.1.1]{wintenberger_corps_1983}}]
    Let $$X_K(K^{\cycl})\coloneqq\varprojlim_n \left(K_n^{\opn{cycl}}\right)^\times\cup\{0\},$$
    where the transition map $\calN_n:\left(K_{n+1}^{\opn{cycl}}\right)^\times\lto \left(K_n^{\opn{cycl}}\right)^\times$ is the norm map. We call it the field of norms of the extension $K^{\opn{cycl}}/K$.
\end{definition}
\begin{theorem}[Fontaine-Wintenberger]\leavevmode
    \begin{enumerate}
        \item (cf. \cite[Th\'eor\`eme 2.1.3]{wintenberger_corps_1983}) By defining [] and [], $X_K(K^{\opn{cycl}})$ is a complete discrete valued field with valuation given by
              $$v((x_n)_{n\geq 1})=[v_p(K^\times):\bbZ]\cdot v_p(x_0).$$
        \item (cf. \cite[Corollaire 3.2.3]{wintenberger_corps_1983}) The absolute Galois group of $X_K(K^{\opn{cycl}})$ is canonically isomorphic to $H_K$.
        \item (cf. \cite[Proposition 4.2.1]{wintenberger_corps_1983}) The field $X_K(K^{\opn{cycl}})$ embeds continuously into the perfectoid field $\widetilde{\bfE}=\bbC_p^\flat\coloneqq \varprojlim_{x\mapsto x^p}\bbC_p$ by the formula [].
        \item The field $X_K(K^{\opn{cycl}})$ is naturally equipped with the action of Frobenius map $x\mapsto x^p$ and the Galois group $\Gamma_K$.
    \end{enumerate}
\end{theorem}
\begin{definition}
    Let $\bfE_K$ be the image of $X_K(K^{\opn{cycl}})$ in $\widetilde{\bfE}$. Set $\bfE=\bigcup_{[K:\bbQ_p]<\infty}\bfE_K$ and $\widetilde{\bfE}_K=\widetilde{\bfE}^{H_K}$.
\end{definition}
\begin{remark}
    With easy calculation, one can see that the element $\varepsilon-1$, where $\varepsilon=(\zeta_{p^n})_{n\geq 1}\in \varprojlim_{x\mapsto x^p}\calO_{\bbC_p}$, is a uniformizer of $\bfE_{\bbQ_p}$.
\end{remark}
Then one has
\begin{proposition}\leavevmode
    \begin{enumerate}
        \item $\bfE$ is the separable closure of $\bfE_{\bbQ_p}$ and is stable under the action of $\scrG_{\bbQ_p}$.
        \item (cf. \cite[Corollaire 4.3.4]{wintenberger_corps_1983}) $\widetilde{\bfE}$ (resp. $\widetilde{\bfE}_K$) is the completed perfect closure of $\bfE$ (resp. $\bfE_K$).
    \end{enumerate}
\end{proposition}

Now we can lift these period rings to characteristic $0$:
\begin{definition}
    Let $\widetilde{\bfA}=W(\widetilde{\bfE})$ be the ring of Witt vectors over $\widetilde{\bfE}$. Let $\bfA_K=\opn{Cohen}(\bfE_K)$ (resp. $\bfA=\opn{Cohen}(\bfE)$) be the Cohen ring\footnote{We refer to \cite[Chapitre IX]{bourbaki_algebre_2006} for the detailed construction of the Wiit vectors and the Cohen ring.} over $\bfE_K$ (resp. $\bfE$). Denote by $\widetilde{\bfB}$ (resp. $\bfB_K$, $\bfB$) the fraction field of $\widetilde{\bfA}$ (resp. $\bfA_K$, $\bfA$).
\end{definition}

We have the following properties of these period rings of characteristic $0$. Most of them follow from the universal properties of the Witt vectors and the Cohen ring:
\begin{proposition}\leavevmode
    \begin{enumerate}
        \item The field $\bfB_K$ is a discrete-valued field with valuation ring $\bfA_K$ and residue field $\bfE_K$.
        \item The field $\bfB$ is the completed maximal unramified extension of $\bfB_{\bbQ_p}$, with ring of integer $\bfA$ and residue field $\bfE$.
        \item One has
              $$\bfA_{\bbQ_p}=\widehat{\bbZ_p\llbracket\pi\rrbracket[\frac{1}{\pi}]},$$
              where $\pi\coloneqq [\varepsilon]-1$ and the hat is taking the $p$-adic completion.
        \item The action of $\scrG_{\bbQ_p}$ on $\bfA_{\bbQ_p}$ is given by
              $$g\cdot\pi=(1+\pi)^{\chi(g)}-1,$$ where $\chi:\scrG_{\bbQ_p}\lto \bbZ_p^\times$ is the cyclotomic character. The action of the Frobenius map $\varphi$ on $\bfA_{\bbQ_p}$ is given by
              $$\varphi(\pi)=(1+\pi)^p-1.$$
        \item One has $\bfA^{H_K}=\bfA_K$, $\bfB^{H_K}=\bfB_K$.
        \item The field $\widetilde{\bfB}$ (resp. $\widetilde{\bfB}_K$) is the completed $\varphi$-perfection of $\bfB$ (resp. $\bfB_K$).
    \end{enumerate}
\end{proposition}
\subsection{\'Etale $(\varphi,\Gamma)$-modules and Fontaine's equivalence of categories}
The strategy of Fontaine to study the $p$-adic representation of $\scrG_K$ is to divide this large group into two parts: $\Gamma_K$ and $H_K$. Then by keeping the action of $\Gamma_K$, which is simply a topologically cyclic group, and replacing the action of $H_K$ with the action of Frobenius map $\varphi$, we can recover the representation of full $\scrG_K$ at the cost of a more complicated coefficient ring.
\begin{definition}[{\cite{Fontaine2007}}]\leavevmode
    \begin{enumerate}
        \item A $\varphi$-module over $\bfA_K$ (resp. $\bfB_K$) is a finitely generated $\bfA_K$-module (resp. $\bfB_K$-vector space) equipped with a semi-linear continuous action of the Frobenius map $\varphi$.
        \item A $\varphi$-module $D$ over $\bfA_K$ is called \'etale if $\varphi(D)$ generates $D$ as an $\bfA_K$-module.
        \item A $\varphi$-module $D$ over $\bfB_K$ is called \'etale if it has an $\bfA_K$-lattice which is \'etale.
    \end{enumerate}
\end{definition}
As we mentioned before, $\varphi$-modules can be used to recover the representation of $H_K$. To be more precise, one has the following result:
\begin{proposition}[{\cite[Proposition 1.2.6]{Fontaine2007}}]\label{thm:44873}
    The functor $\calD$:
    $$\{\text{finite dimensional }\bbZ_p\text{-representations of }H_K\}\xlto{\calD} \{\text{\'etale } \varphi\text{-modules over }\bfA_K\},$$
    $$V\longmapsto (\bfA\otimes_{\bbZ_p} V)^{H_K};$$
    is an equivalence of categories, with quasi-inverse $\calV$ given by $\calV(D)\coloneqq (\bfA\otimes_{\bfA_K} D)^{\varphi=1}$.
\end{proposition}
\begin{remark}\label{rmk:44018}
    By replacing $\bfA$ (resp. $\bfA_K$) by $\bfB$ (resp. $\bfB_K$), one can also define the equivalence of categories between finite-dimensional $\bbQ_p$-representations of $\scrG_K$ and \'etale $\varphi$-modules over $\bfB_K$.
\end{remark}
\begin{definition}[{\cite{Fontaine2007}}]
    An \'etale $(\varphi,\Gamma)$-module over $\bfA_K$ (resp. $\bfB_K$) is an \'etale $\varphi$-module over $\bfA_K$ (resp. $\bfB_K$) with a semi-linear action of $\Gamma_K$ which commutes with the semi-linear action of $\varphi$.
\end{definition}
By \Cref{thm:44873}, it is not hard to get the following important result:
\begin{theorem}[{\cite[Th\'eor\`eme A.3.4.3]{Fontaine2007}}]\label{thm:24308}
    The functor $\calD$ in \Cref{thm:44873} induces an equivalence of categories between finite-dimensional $\bbZ_p$- (resp. $\bbQ_p$-)representations of $\scrG_K$ and \'etale $(\varphi,\Gamma)$-modules over $\bfA_K$ (resp. $\bfB_K$). Its quasi-inverse is given by $\calV$ in \Cref{thm:44873}.
\end{theorem}

\subsection{Herr complex and Galois cohomology}
Since we have established the equivalence of categories in \Cref{thm:24308}, it is natural to use the \'etale $(\varphi,\Gamma)$-modules to recover the information of the $p$-adic Galois representations. For example, we can use \'etale $(\varphi,\Gamma)$-modules to calculate the Galois cohomology. This is discovered by Herr:
\begin{theorem}[{\cite[Th\'eor\`eme 2.1]{herr_sur_1998}}]\label{thm:12830}
    Let $\gamma$ be a topological generator of $\Gamma_K$. For any \'etale $(\varphi,\Gamma)$-modules $D$ over $\bfA_{K}$ (resp. $\bfB_K$). Define the complex
    $$\calC_{\varphi,\gamma}(D):\begin{tikzcd}[sep=huge, ampersand replacement=\&]
            0\ar[r]\&D\ar[r,"\begin{pmatrix}
                    \varphi-1 \\\gamma-1
                \end{pmatrix}"]\&D\oplus D\ar[r,"{\begin{pmatrix}
                \gamma-1 & 1-\varphi
            \end{pmatrix}}"]\& D\ar[r]\&0
        \end{tikzcd}.$$
    Then for any $\bbZ_p$- (resp. $\bbQ_p$-)representation $V$ of $\scrG_K$, one has isomorphism of cohomology groups:
    $$H^i(\scrG_K,V)\cong H^i(\calC_{\varphi,\gamma}(\calD(V)),\ i=0,1,2.$$
\end{theorem}

\subsection{The $\psi$-operator}
The following lemma follows from \cite[Theorem 29.2]{matsumura_commutative_1987} and a devissage argument:
\begin{lemma}
    One has
    $$[\bfA:\varphi(\bfA)]=[\bfA_K:\varphi(\bfA_K)]=p.$$
    Moreover, one can take a common basis of these two extensions:
    $1,[\varepsilon],[\varepsilon]^2,\cdots,[\varepsilon]^{p-1}$.
\end{lemma}
\begin{definition}
    The $\psi$-operator on the period ring $\bfA$ is defined as:
    $$\psi:\bfA\lto\bfA,\ x\longmapsto \frac{1}{p}\varphi^{-1}(\opn{tr}_{\bfA/\varphi(\bfA)}(x)).$$
    By restricting this operator to $\bfA_K$, we get the $\psi$-operator on $\bfA_K$.
\end{definition}
One can easily verify the following properties of the $\psi$-operator:
\begin{lemma}\leavevmode
    \begin{enumerate}
        \item The $\psi$-operator is a left-inverse of $\varphi$, i.e. $\psi\circ\varphi=\opn{id}$.
        \item For $x\in\bfA$, if we write $x=\sum_{k=0}^{p-1}\varphi(x_k)[\varepsilon]^k$, where $x_k\in\bfA$, $k=0,1,\cdots,p-1$, then $\psi(x)=x_0$.
        \item $\psi$ commutes with $\scrG_{\bbQ_p}$.
    \end{enumerate}
\end{lemma}

It is natural to generalize the $\psi$-operator to \'etale $(\varphi,\Gamma)$-modules over $\bfA_K$:
\begin{lemma}
    If $D$ is an \'etale $(\varphi,\Gamma)$-modules over $\bfA_K$, there exists a unique operator $\psi:D\lto D$, satisfying that for every $a\in\bfA_K$ and $x\in D$,
    $$\psi(\varphi(a)x)=a\psi(x),\ \psi(a\varphi(x))=\psi(a)x.$$ Moreover, $\psi$ commutes with $\Gamma_K$.
\end{lemma}
\begin{proof}
    Define $\psi$ to be the operator $\psi\otimes 1$ on $D\subset\bfA\otimes_{A_K}\calV(D)$.  The uniqueness follows from the \'etaleness of $D$.
\end{proof}

\begin{corollary}
    Let $D$ be an \'etale $(\varphi,\Gamma)$-module over $\bfA_K$. Then for any $x\in D$ and $n\geq 1$, one has
    $$x=\sum_{i=0}^{p^n-1}[\varepsilon]^i\varphi^n(x_i),$$
    where $x_i=\psi^n([\varepsilon]^{-i}x)$. It is easy to see if $F(\pi)$ belongs to $\bbZ_p\llbracket\pi\rrbracket$, $F_i(\pi)$ belongs to $\bbZ_p\llbracket\pi\rrbracket$ for all $i$. Then $\psi(\bbF_p\llbracket\bar{\pi}\rrbracket)\subset\bbF_p\llbracket\bar{\pi}\rrbracket$, hence $\psi(\bbZ_p\llbracket\pi\rrbracket)\subset\bbZ_p\llbracket\pi\rrbracket$.
\end{corollary}

\begin{example}\label{eg:57892}
    Let $D=\bfA_{\bbQ_p}\supset \bbZ_p\llbracket \pi\rrbracket$ be the trivial $(\varphi,\Gamma)$-module. Then for $x=F(\pi)\in\bbZ_p\llbracket\pi\rrbracket$, $\varphi(x)=F((1+\pi)^p-1)$. Write
    $$F(\pi)=\sum_{i=0}^{p-1}(1+\pi)^i F_i((1+\pi)^p-1),$$
    then $\psi(F(\pi))=F_0(\pi)$. Consequently, $\psi$ is continuous for the weak topology.

    Moreover, we have
    \begin{equation}
        \begin{aligned}
            \varphi(\psi(x))= & F_0((1+\pi)^p-1)                                                                  \\
            =                 & \sum_{i=0}^{p-1}\left(\frac{1}{p}\sum_{z^p=1}z^i\right)(1+\pi)^i F_i((1+\pi)^p-1) \\
            =                 & \frac{1}{p}\sum_{z^p=1}\sum_{i=0}^{p-1}(z(1+\pi))^i F_i((z(1+\pi))^p-1)           \\
            =                 & \frac{1}{p}\sum_{z^p=1}F(z(1+\pi)-1).
        \end{aligned}
    \end{equation}
    This shows that this $\psi$-operator is the same one as in \cite[Proposition 2.2.3]{coates_cyclotomic_2006}.
\end{example}


\section{Iwasawa cohomology of $p$-adic representations}
\subsection{Preliminaries on Iwasawa algebra}
For any $n\geq $, we take a topological generator $\gamma_n$ of the Galois group $\Gamma_n=\opn{Gal}(K^{\opn{cycl}}/K_n^{\opn{cycl}})$, satisfying $\gamma_n=\gamma_1^{p^{n-1}}$. The Iwasawa algebra
$$\bbZ_p\llbracket\Gamma_K\rrbracket\coloneqq \varprojlim_n \bbZ_p[\Gamma_K/\Gamma_n]$$
is isomorphic to $\bbZ_p\llbracket T\rrbracket$ with the $(p,T)$-adic topology by sending $T$ to $\gamma-1$. We have
$$\bbZ_p\llbracket \Gamma_K\rrbracket/(\gamma_n-1)=\bbZ_p[\opn{Gal}(K_n^{\opn{cycl}}/K)].$$
Furthermore, $\scrG_K$ acts on $\bbZ_p\llbracket\Gamma_K\rrbracket$ and $\bbZ_p[\opn{Gal}(K_n^{\opn{cycl}}/K)]$ through $\Gamma_K$.


\subsection{Iwasawa cohomology}
\begin{definition}[{\cite{perrin-riou_theorie_1994}, \cite[D\'efinition II.1.1]{cherbonnier_theorie_1999}}]\leavevmode
    \begin{enumerate}
        \item If $V$ is a $\bbZ_p$-representation of $\scrG_K$, we define the Iwasawa cohomology of $V$ to be
              $$H^i_{\opn{Iw}}(K,V)\coloneqq \varprojlim_n H^i(\scrG_{K_n^{\opn{cycl}}},V),$$
              where the transition maps are the corestriction maps.
        \item If $V$ is a $\bbQ_p$-representation, we define the Iwasawa cohomology of $V$ to be
              $$H^i_{\opn{Iw}}(K,V)\coloneqq H^i_{\opn{Iw}}(K,T)\otimes_{\bbZ_p} \bbQ,$$
              where $T$ is arbitary $\scrG_K$-stable $\bbZ_p$-lattice in $V$.
    \end{enumerate}
\end{definition}

\begin{proposition}
    For a $\bbZ_p$- (resp. $\bbQ_p$-)representation of $\scrG_K$, one has an isomorphism
    $$H^i(\scrG_K,\bbZ_p\llbracket \Gamma_K\rrbracket\otimes V)\cong H^i_{\opn{Iw}}(K,V).$$
\end{proposition}
\begin{proof}
    It reduces to the case that $V$ is a $\bbZ_p$-representation of $\scrG_K$.

    By using Shapiro lemma and viewing $V$ as a $\bbZ_p[\scrG_K]$-module, we have an isomorphism
    $$H^i(\scrG_{K_n^{\opn{cycl}}},V)\cong H^i(\scrG_K,\bbZ_p[\opn{Gal}(K_n^{\opn{cycl}}/K)]\otimes V).$$
    Thus
    $$H^i_{\opn{Iw}}(K,V)\cong\varprojlim_n H^i(\scrG_K,\bbZ_p[\opn{Gal}(K_n^{\opn{cycl}}/K)]\otimes V)=\varprojlim_n H^i(\scrG_K,\bbZ_p\llbracket\Gamma_K\rrbracket/(\gamma_n-1)\otimes V).$$
\end{proof}

The main theorem of this section is the following:
\begin{theorem}[{\cite[Th\'eor\`eme II.1.3, Remarque II.3.2]{cherbonnier_theorie_1999}}]\label{thm:15019}
    If $V$ is a $\bbZ_p$ or $\bbQ_p$ representation of $\scrG_K$, then we have:
    \begin{enumerate}
        \item $H^i_{\opn{Iw}}(K,V)=0$, if $i\neq 1,2$.
        \item There exist canonical isomorphisms
              \begin{equation}\label{eq:43260}
                  \calD(V)^{\psi=1}\xlto{\cong}H^1_{\opn{Iw}}(K,V),
              \end{equation}
              $$\calD(V)/(\psi-1)\xlto{\cong}H^2_{\opn{Iw}}(K,V).$$
    \end{enumerate}
\end{theorem}
\begin{proof}[Sketch of proof]\leavevmode
    The theorem is trivial when $i\geq 3$, so we focus on the cases of $i=0,1,2$. Set $D=\calD(V)$.
    \begin{enumerate}[label=\textit{Step \arabic*.},wide, labelwidth=!, labelindent=0pt]
        \item \textit{Calculating Galois cohomology by $\psi$-complex.}

              By replacing $\varphi$ in Herr's complex with $\psi$, we obtain a new complex $\calC_{\psi,\gamma}(D)$ with a morphism:
              $$\begin{tikzcd}[sep=huge, ampersand replacement=\&]
                      \calC_{\varphi,\gamma}(D):\&0\ar[r]\&D\ar[r]\ar[d,equal]\&D\oplus D\ar[r]\ar[d,"{\begin{psmallmatrix}
                          -\psi & 0\\0&\opn{Id}
                      \end{psmallmatrix}}"]\& D\ar[r]\ar[d,"-\psi"]\&0\\
                      \calC_{\psi,\gamma}(D):\&0\ar[r]\&D\ar[r,"\begin{psmallmatrix}
                              \psi-1 \\\gamma-1
                          \end{psmallmatrix}"']\&D\oplus D\ar[r,"{\begin{psmallmatrix}
                          \gamma-1 & 1-\psi
                      \end{psmallmatrix}}"']\& D\ar[r]\&0
                  \end{tikzcd}.$$
              Since $\psi$ is surjective on $D$, the above morphism of complexes is also surjective.
              One can show that the kernel complex of the above morphism
              $$\begin{tikzcd}
                      0\ar[r]&D^{\psi=0}\ar[r,"\gamma-1"]&D^{\psi=0}\ar[r]&0
                  \end{tikzcd}$$
              is exact, i.e. $\gamma-1$ is invertible on $D^{\psi=0}$. This turns out to be the most delicate point of all the computations and is proved in \cite[Th\'eor\`eme 3.8]{herr_sur_1998}.

              Thus $\calC_{\psi,\gamma}(D)$ is quasi-isomorphic to $\calC_{\varphi,\gamma}(D)$ and consequently calculates the Galois cohomology $H^i(\scrG_K,V)$ by Herr's theorem (cf. \Cref{thm:12830}).
        \item \textit{Taking inverse limit.}

              The result in Step 1 has several quick consequences:
              \begin{enumerate}
                  \item $H^0(\scrG_K,V)\cong \calD(V)^{\psi=1,\gamma=1}$;
                  \item $H^2(\scrG,V)\cong \frac{\calD(V)}{\psi-1}/(\gamma-1)$;
                  \item (cf. \cite[Lemme I.5.5]{cherbonnier_theorie_1999}) There is an exact sequence
                        \begin{equation}\label{eq:57298}\begin{tikzcd}
                                0\ar[r]&\frac{\calD(V)^{\psi=1}}{\gamma-1}\ar[r]&H^1(\scrG,V)\ar[r]&\left(\frac{\calD(V)}{\psi-1}\right)^{\gamma=1}\ar[r]&0\\
                                &&(x,y)\ar[r,maps to]&x&
                            \end{tikzcd}.\end{equation}
              \end{enumerate}
              For $i=0$,
              $$H^0(K,V)=\varprojlim_{\opn{Tr}}V^{\scrG_{K_n}^{\opn{cycl}}},$$
              where $\opn{Tr}$ is the trace map. Since $V^{\scrG_{K_n^{\opn{cycl}}}}$ is increasing and $V$ is of finite-dimensional, the sequence is stationary for $n>n_0$, where $n_0>\!\!>0$. Then $\opn{Tr}_{K_{n+1}^{\opn{\cycl}}/K_n^{\opn{cycl}}}$ is just multiplication by $p$ for $n\geq n_0$. Since $V$ does not contain $p$-divisible elements, this shows that $\varprojlim_{\opn{Tr}}V^{\scrG_{K_n^{\opn{cycl}}}}=0$.

              For $i=2$, by replacing $K$ with $K_n^{\opn{cycl}}$, one has an isomorphism $$H^2(\scrG_{K_n^{\opn{cycl}}},V)\cong \frac{\calD(V)}{\psi-1}/(\gamma_n-1).$$
              With this identification, the corestriction map is induced by
              the identity map $\opn{Id}$ on $\calD(V)$. Since $\frac{\calD(V)}{\psi-1}$ is compact (cf. \cite[Proposition I.5.6]{cherbonnier_theorie_1999}), one concludes that
              $$H^2_{\opn{Iw}}(K,V)=\varprojlim_{n}\frac{\calD(V)}{\psi-1}/(\gamma_n-1)=\frac{\calD(V)}{\psi-1},$$
              where the last equality is \cite[Proposition II.3.1]{cherbonnier_theorie_1999}.

              The case of $i=1$ is of most interest. It is easy to check that the exact sequence \Cref{eq:57298} admits the following naturality:
              $$\begin{tikzcd}
                      0\ar[r]&\frac{\calD(V)^{\psi=1}}{\gamma_n-1}\ar[r]\ar[d]&H^1(\calG_{K_n^{\opn{cycl}}},V)\ar[r]\ar[d,"\opn{cores}"]&\left(\frac{\calD(V)}{\psi-1}\right)^{\gamma_n-1}\ar[r]\ar[d,"\frac{\gamma_n-1}{\gamma_{n-1}-1}"]&0\\
                      0\ar[r]&\frac{\calD(V)^{\psi=1}}{\gamma_{n-1}-1}\ar[r]&H^1(\calG_{K_{n-1}^{\opn{cycl}}},V)\ar[r]&\left(\frac{\calD(V)}{\psi-1}\right)^{\gamma_{n-1}-1}\ar[r]&0
                  \end{tikzcd}.$$
              Thus by taking the inverse limit on \Cref{eq:57298}, we get the exact sequence
              $$\begin{tikzcd}
                      0\ar[r]&\varprojlim_n\frac{\calD(V)^{\psi-1}}{\gamma_n-1}\ar[r]&\varprojlim_n H^1(\scrG_{K_n^{\opn{cycl}}},V)\ar[r]&\varprojlim_n\left(\frac{\calD(V)}{\psi-1}\right)^{\gamma_n=1}\ar[r]&0
                  \end{tikzcd}.$$
              Since $\calD(V)^{\psi=1}$ is compact, by \cite[Proposition I.5.6]{cherbonnier_theorie_1999} again we have $\calD(V)^{\psi=1}\cong\varprojlim_n \frac{\calD(V)^{\psi=1}}{\gamma_n-1}$. On the other hand, similar argument for showing $H^0_{\opn{Iw}}(K,V)=0$ shows that $\varprojlim_n\left(\frac{\calD(V)}{\psi-1}\right)^{\gamma_n=1}=0$, which implies that $\calD(V)^{\psi=1}\cong H^1_{\opn{Iw}}(K,V)$.
    \end{enumerate}
\end{proof}

The inverse of the isomorphism \Cref{eq:43260} in \Cref{thm:15019}, which we denote\footnote{We call it the exponential map, for it is closed related to the Bloch-Kato's exponential map for de Rham representations.} by $\opn{Exp}^*_{V^*(1)}$, is the map that will produce $p$-adic L-functions in the next section. However, it is not easy to write down an explicit formula for this map. On the other hand, Chebrbonnier and Colmez gave an explicit formula for the isomorphism \Cref{eq:43260}, which we denote by $\opn{Log}^*_{V^*(1)}$:
\begin{theorem}[{\cite[Section I.5, Th\'eor\`eme II.1.3]{cherbonnier_theorie_1999}}]\leavevmode
    \begin{enumerate}
        \item For $n\geq 1$ and $y\in\calD(V)^{\psi=1}$, there exists $b_n(y)\in \bfA\otimes V$ such that $$(\gamma_n-1)(\varphi-1)b_n(y)=(\varphi-1)y.$$
        \item For $n\geq 1$ and $y\in\calD(V)^{\psi=1}$, the map
              $$\sigma\longmapsto \frac{\sigma-1}{\gamma_n-1}y-(\sigma-1)b_n(y)$$
              is a cocycle in $H^1(K_n^{\opn{cycl}},V)$.
        \item The map
              $$h_{n,V}^1:\calD(V)^{\psi=1}\lto H^1(K_n^{\opn{cycl}},V),\ y\longmapsto \frac{\log(\chi(\gamma_n))}{p^n}\left(\sigma\longmapsto \frac{\sigma-1}{\gamma_n-1}y-(\sigma-1)b_n(y)\right)$$
              does not depend on the choice of generator $\gamma_n$.
        \item The map $$\opn{Log}^*_{V^*(1)}\colon y\longmapsto \varprojlim_n h_{n,V}^1(y)$$ is an isomorphism from $\calD(V)^{\psi=1}$ to $H_{\opn{Iw}}^1(K,V)$.
    \end{enumerate}
\end{theorem}
\section{The explicit reciprocity law: a baby case}
\subsection{The module $\calD(\bbZ_p(1))^{\psi=1}$ and Coleman's power series}
The module $\bbZ_p(1)$ is just $\bbZ_p$ with the action of $\scrG_{\bbQ_p}$ given by $g\cdot x=\chi(g)x$ for every $g\in\scrG_{\bbQ_p}$ and $x\in\bbZ_p(1)$. One has
$$\calD(\bbZ_p(1))=\left(\bfA\otimes\bbZ_p(1)\right)^{H_{\bbQ_p}}=\bfA_{\bbQ_p}(1),$$
with usual actions of $\varphi$ and $\psi$ and for $\gamma\in\Gamma$, it acts on $\bfA_{\bbQ_p}$ by $\gamma\cdot f(\pi)=\chi(\gamma)f((1+\pi)^{\chi(\gamma)}-1)$.

\begin{proposition}\leavevmode
    \begin{enumerate}
        \item One has isomorphism $\bfA_{\bbQ_p}^{\psi=1}=\bbZ_p\cdot \frac{1}{\pi}\oplus \left(\bfA_{\bbQ_p}^+\right)^{\psi=1}$;
        \item There exists an exact sequence
              $$\begin{tikzcd}
                      0\ar[r]&\bbZ_p\ar[r]&\left(\bfA_{\bbQ_p}^+\right)^{\psi=1}\ar[r]&\left(\pi\bfA_{\bbQ_p}^+\right)^{\psi=0}\ar[r]&0
                  \end{tikzcd}$$
    \end{enumerate}
\end{proposition}
\begin{proof}
    [TBA]
\end{proof}

\begin{remark}
    Under the map $\mu\longmapsto\int_{\bbZ_p}[\varepsilon]^x\mu$, $\left(\pi\bfA_{\bbQ_p}^+\right)^{\psi=0}$ is the image of measures with support in $\bbZ_p^\times$ (for $\psi=0$) and $\int_{\bbZ_p^\times}\mu=0$.
\end{remark}

Besides that, let us recall two results that we have learned in the lesson.
\begin{theorem}[Coleman's power series]\leavevmode\begin{enumerate}
        \item (cf. \cite[Theorem 2.1.2]{coates_cyclotomic_2006}) Let $u=(u_n)_{n\geq 1}\in\varprojlim_n \calO_{\bbQ_{p,n}^{\opn{cycl}}}^\times$, then there exists a unique power series $f_u\in\bbZ_p\llbracket T\rrbracket$ such that $f_u(\pi_n)=u_n$ for all $n\geq 1$.
        \item (cf. \cite[Lemma 2.4.5]{coates_cyclotomic_2006}, \Cref{eg:57892}) Let $\partial=(1+T)\frac{\dif}{\dif T}$. Then $\frac{\partial f_u}{f_u}(\pi)\in\left(\bfA_{\bbQ_p}^+\right)^{\psi=1}$.
    \end{enumerate}

\end{theorem}
\subsection{Input from Kummer theory}
Recall that we set $\varepsilon=(1,\zeta_p,\cdots,\zeta_{p^n},\cdots)\in\bfE_{\bbQ_p}$. If for every $n\geq 1$, we set $\pi_n=\zeta_{p^n}-1$, then $\pi_n$ is a uniformizer of $\bbQ_{p,n}^{\opn{cycl}}$, and $\calN_{\bbQ_{p,n+1}^{\opn{cycl}}/\bbQ_{p,n}^{\opn{cycl}}}(\pi_{n+1})=\pi_n$.

For an element $a$ of $\bbQ_{p,n}^{\opn{cycl},\times}$, choose $x=(a,x^{(1)},\cdots,)\in\bbC_p^\flat$. This $x$ is uniquely determined by $a$ up to $\varepsilon^u$ with $u\in\bbZ_p$. Therefore if $g\in\scrG_{\bbQ_{p,n}^{\opn{cycl}}}$, then
$g(x)/x=\varepsilon^{c(g)}$, with $c(g)\in\bbZ_p$. This gives a $1$-cocycle $c$ on $\scrG_{\bbQ_{p,n}^{\opn{cycl}}}$ with values in $\bbZ_p(1)$ and defines the Kummer map
$$\kappa\colon \bbQ_{p,n}^{\opn{cycl},\times}\lto H^1\left(\scrG_{\bbQ_{p,n}^{\opn{cycl}}},\bbZ_p(1)\right).$$
By Kummer theory, we have
$$H^1(\scrG_{\bbQ_{p,n}^{\opn{cycl}}},\bbZ_p(1))\cong\bbZ_p\cdot\kappa(\pi_n)\oplus \kappa\left(\calO_{\bbQ_{p,n}^{\opn{cycl}}}\right).$$
On the other hand, one can verify that the diagram
$$\begin{tikzcd}
        \bbQ_{p,n}^{\opn{cycl},\times}\ar[r,"\kappa"]\ar[d,"\calN_{\bbQ_{p,n+1}^{\opn{cycl}}/\bbQ_{p,n+1}^{\opn{cycl}}}"]&H^1\left(\scrG_{\bbQ_{p,n+1}^{\opn{cycl}}},\bbZ_p(1)\right)\ar[d,"\opn{cores}"]\\
        \bbQ_{p,n}^{\opn{cycl},\times}\ar[r,"\kappa"]&H^1\left(\scrG_{\bbQ_{p,n}^{\opn{cycl}}},\bbZ_p(1)\right)
    \end{tikzcd}$$
commutes, which induces a map
$$\kappa\colon \varprojlim_n \bbQ_{p,n}^{\opn{cycl},\times}\lto H_{\opn{Iw}}^1(\bbQ_p,\bbZ_p(1))$$
and consequently an isomorphism
$$H_{\opn{Iw}}^1(\bbQ_p,\bbZ(1))=\bbZ_p\cdot\kappa(\pi_n)\oplus\kappa\left(\varprojlim_n\bbQ_{p,n}^{\opn{cycl},\times}\right).$$

\subsection{Main theorem}
The main theorem of this thesis is the following:
\begin{theorem}[Explicit reciprocity law]
    The following diagram is commutative:
    $$\begin{tikzcd}
            \varprojlim_n \bbQ_{p,n}^{\opn{cycl},\times}\ar[rr,"\kappa"]\ar[dr,"u\mapsto \frac{\partial f_u}{f_u}(\pi)"']&&H_{\opn{Iw}}^1(\bbQ_p,\bbZ_p(1))\ar[dl,"\opn{Exp}^*","\cong"']\\
            &\calD(\bbZ_p(1))^{\psi=1}&
        \end{tikzcd}.$$
\end{theorem}

\begin{remark}
    Let $a\geq 2$ be an integer that coprime to $p$. Recall that the element
    $$\bfc_n(a)=\frac{\exp(-a\cdot2\pi i/p^n)-1}{\exp(-2\pi i/p^n)-1}\in\bbQ(\zeta_{p^n})$$
    is a cyclotomic unit in $\calO_{\bbQ_(\zeta_{p^n})}$ and the series $\bfc(a)=(u_n)_{n\geq 1}$ lies in $\varprojlim_n \calO_{\bbQ_{p,n}^{\opn{cycl}}}^\times$. It is easy to verify that the Coleman power series of $\bfc(a)$ is given by
    $$f_{\bfc(a)}=\frac{(1+T)^{-a}-1}{(1+T)^{-1}-1},$$
    which satisfies
    $$\frac{\partial f_{\bfc(a)}}{f_{\bfc(a)}}=\frac{a}{(1+T)^a-1}-\frac{1}{T}.$$
    This is the Amice transform of the measure $\mu_a$ we used to construct the Kubota-Leopoldt zeta function (cf. \cite[Proposition 1.6]{Col}). Therefore the reciprocity law shows that $\opn{Exp}*$ products Kubota-Leopoldt zeta function from the system of cyclotomic units.
\end{remark}
\begin{remark}
    The example in [] is part of a big conjectural picture. For $V$ a fixed representation of $\scrG_\bbQ$, then conjecturally
    $$\{\text{compatible system of global elements of $V$}\}\lto H_{\opn{Iw}}^1(\bbQ,V)$$
    $$\lto H_{\opn{Iw}}^1(\bbQ_p,V)\xlto{\opn{Exp}^*}\calD(V)^{\psi=1}\xlto[\text{transform}]{\text{Amice}} p\text{-adic $L$-functions}.$$
\end{remark}
\printbibliography
\end{document}